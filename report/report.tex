% !TEX program = lualatex
% !TEX encoding = UTF-8 Unicode
% 
\newcommand{\titol}{Sistemas basados en el conocimiento}
\newcommand{\materia}{Inteligencia artificial}
\newcommand{\idioma}{english,spanish}
\newcommand{\pdfauthors}{Nil Mamano Grande, Héctor Ramón Jiménez, Isaac Sánchez Barrera}
%\newcommand{\autors}[1]{\begin{tabular}{#1} Nil Mamano Grande \\ Héctor Ramón Jiménez \\ Isaac Sánchez Barrera\end{tabular}}
\newcommand{\autors}[1]{\begin{tabular}{#1} Mamano -- Ramón -- Sánchez\end{tabular}}
\newcommand{\data}{\today}
\include{header}
\setstretch{1.0}
\DefineBibliographyStrings{spanish}{%
  references = {Referencias},
}

\title{\materia\\
\Large{\titol}}
\subtitle{Facultat d'Informàtica de Barcelona\\ % Pongo la I mayúscula porque la FIB lo hace
Universitat Politècnica de Catalunya}
\author{
  Nil Mamano Grande \\
  Héctor Ramón Jiménez \\
  Isaac Sánchez Barrera}
\date{
  \today \\
  cuatrimestre de otoño \\
  curso 2013--2014}

\everymath{\displaystyle}

\newcommand{\CC}{\mathbb{C}}
\newcommand{\RR}{\mathbb{R}}
\newcommand{\NN}{\mathbb{N}}
\newcommand{\bigO}[1]{\ensuremath{\operatorname{O}\left(#1\right)}}% big-O notation/symbol
\newcommand{\bigOmega}[1]{\ensuremath{\operatorname{\Omega}\left(#1\right)}}% big-O notation/symbol
\newcommand{\bigTheta}[1]{\ensuremath{\operatorname{\Theta}\left(#1\right)}}% big-O notation/symbol
\newcommand{\slot}[1]{\textsl{\texttt{#1}}}
\newcommand{\clase}[1]{\texttt{#1}}

\newenvironment{slotlist}{%
   \renewcommand\descriptionlabel[1]{\hspace{\labelsep}\slot{##1}}
   \begin{description}%
}{%
   \end{description}%
}

\bibliography{references}
\begin{document}

\maketitle
\tableofcontents
\listoftables
\listoffigures
\vfill
\cleardoublepage

\part{Presentación y análisis del problema}
% !TEX encoding = UTF-8 Unicode
% !TEX root = ../report.tex
% 

\section{Introducción}
\subsection{Descripción del problema}

\begin{wrapfigure}{O}{0.3\textwidth}
  \vspace{-20pt}
  \begin{center}
    \includegraphics[width=0.28\textwidth]{figures/ricorico}
  \end{center}
  \vspace{-20pt}
\end{wrapfigure}

La empresa de catering \emph{Rico Rico} quiere mejorar su eficiencia a la hora
de proponer menús para las celebraciones de sus clientes. Con ese fin, ha
encargado la implementación de un sistema experto basado en el conocimiento que
tienen tras su larga experiencia en el sector. Gracias al sistema, solamente es
necesario que el cliente indique sus preferencias y restricciones concretas
para obtener las recomendaciones más acordes con sus necesidades.

El sistema se encarga de hacer preguntas al cliente para recomendar hasta tres
menús con distintos precios, siempre dentro de los límites exigidos por el
cliente. Entre otras cosas, se precisa saber el tipo de evento, el número de
comensales, los posibles trastornos o preferencias alimenticias (alergias,
razones culturales, etc.), preferencias de origen cultural de los platos...

También debe ser capaz de recomendar vinos si el cliente así lo desea, y hacer
un menú lo suficientemente acorde con los éstos. Inicialmente se tenía previsto
que las bebidas también dependieran de los platos y el cliente, pero fuera del
terreno del vino los comensales prefieren escoger individualmente la bebida.

% !TEX encoding = UTF-8 Unicode
% !TEX root = ../report.tex
% 

\section{Conceptualización del problema}
\subsection{Conceptos principales del dominio}

Después de haber hecho un trabajo iterativo, los conceptos que se modelan en el
dominio de conocimiento son
\begin{enumerate}
  \item Cosas elaboradas, como platos y vino (tienen su nombre y precio).
  \item Dentro de los platos, si son pesados o ligeros y su dificultad de
    preparación.
  \item Los ingredientes que forman parte de los platos y su disponibilidad
    durante las cuatro estaciones del año.
  \item Grandes grupos de comensales, que modelan el tipo de ingredientes que
    no pueden comer.
  \item Los tipos de platos, que contienen información sobre los vinos que
    pueden ir mejor con ellos.
  \item Los eventos a celebrar, que tienen los platos que son propios (o
    recomendables) y la importancia de éstos últimos a la hora de elaborar el
    los menús para el cliente.
  \item Las regiones de procedencia de los platos.
  \item El estilo de los platos. Están los genéricos, que no llegarían a ser
    tradicionales porque son simples platos, los tradicionales (podrían
    considerarse, en parte, folclóricos), y platos modernos. Además, también se
    hace distinción de los platos para sibaritas, que son para los paladares
    más finos, pero que pueden pertenecer a cualquiera de las categorías
    anteriores.
\end{enumerate}

Como hemos indicado anteriormente, los comensales que no beben vino acostumbran
a preferir una bebida concreta de forma individualizada. Por esta razón, hemos
pensado que no es algo a modelar en nuestro dominio de conocimiento.

Por otro lado, en el dominio de solución disponemos de
\begin{enumerate}
  \item Recomendaciones concretas de platos, con los motivos para su
    recomendación y una valoración.
  \item Posibles menús abstractos, que contienen el orden de platos y los
    colores de los vinos más aptos para éstos.
  \item Los menús finales con los platos y vinos concretos, además de las
    razones y valoración de su recomendación.
\end{enumerate}

\vfill
\clearpage
\part{Implementación del sistema experto}
% !TEX encoding = UTF-8 Unicode
% !TEX root = ../report.tex
% 

\section{Construcción de la ontología}

\begin{figure}[h!]
  \makebox[\textwidth][c]{\includegraphics[width=1.2\textwidth]{%
      figures/ontologia}}
  \caption{La ontología final}
\end{figure}

% !TEX encoding = UTF-8 Unicode
% !TEX root = ../report.tex
% 

\section{Formalización e implementación del sistema experto mediante
  \texttt{CLIPS}}

\texttt{CLIPS} es una herramienta utilizada para el desarrollo y la ejecución
de sistemas expertos basados en \strong{reglas}.  Estas \strong{reglas} se
aplican cuando ciertos \strong{hechos} ocurren, generando nuevos
\strong{hechos} o modificando algunos ya existentes, que a su vez podrían
disparar otras \strong{reglas}. Esto resulta muy útil para desarrollar sistemas
expertos, pues nos permite expresar con gran facilidad las \strong{reglas de
  inferencia} desde un \strong{conocimiento} hasta otro.  Además,
\texttt{CLIPS} permite estructurar las distintas reglas en \strong{módulos} y
así poder \strong{enfocar} o \strong{activar} un conjunto de reglas
determinado.

\subsection{La regla \regla{inicial}}
La regla \regla{inicial} no tiene precondición (se ejecuta siempre al inicio), muestra un mensaje de bienvenida al cliente y activa el
\strong{módulo de abstracción de datos}.

% !TEX encoding = UTF-8 Unicode
% !TEX root = ../../report.tex
% 

\subsection{Módulo de abstracción de datos}
El módulo de abstracción de datos se encarga de obtener las \strong{preferencias del cliente} y abstraerlas para que el sistema experto
pueda utilizarlas. Este módulo es el primero en ejecutarse, ya que es necesario conocer primero las preferencias del cliente
antes de recomendarle nada.

\subsubsection{Los datos}
Veamos con detalle las interacciones necesarias por parte del cliente para introducir los datos que le pide el módulo de abstracción
y que el experto desarrollado tendrá en cuenta para la recomendación:

\begin{description}
\item[Evento] Indicar de entre los tipos de evento disponibles, aquel que desee realizar.
\item[Número de comensales] Introducir el número de comensales del evento.
\item[Tipos de cocina preferidos] Escoger de entre los tipos de cocina disponibles, aquellos que prefiere sobre los demás.
\item[Sibarita] Indicar si es sibarita y prefiere platos para los paladares más exigentes.
\item[Regiones preferidas] Indicar si desea dar prioridad a los platos de una zona geográfica concreta y, en caso afirmativo, seleccionar
las zonas preferidas de entre las disponibles.
\item[Tipos de comensal] Seleccionar los tipos de comensales que asistirán al evento (vegetarianos, alérgicos, etc.).
\item[Temperatura preferida] Indicar si prefiere platos en una temperatura concreta y, en caso afirmativo, indicar la temperatura preferida de entre las disponibles.
\item[Estación del año] Seleccionar la estación del año en la que se celebrará el evento.
\item[Ingredientes prohibidos] Indicar si desea prohibir algunos ingredientes en concreto y, en caso afirmativo, seleccionar los
ingredientes prohibidos de entre los disponibles.
\item[Limitaciones de precio] Indicar si desea establecer límites de precio y, en caso afirmativo, introducir el precio máximo y
mínimo para los menús recomendados.
\item[Bebida] Escoger la bebida que desea: un vino para el menú, un vino para cada plato o otra bebida.
\item[Colores de vino preferidos] En caso de que se haya preferido vino para la bebida, seleccionar los colores de vino preferidos de
entre los disponibles.
\end{description}

\subsection{Abstracción}
Para abstraer los datos y representarlos en \texttt{CLIPS} hemos utilizado dos \texttt{template}s:

\begin{description}
\item[Preferencias] Cada uno de sus \texttt{slots} representa una preferencia del cliente. Lo más destacable es que se ha utilizado un
 \texttt{multislot} \texttt{ingredientes-prohibidos} que se llena durante la abstracción con aquellos ingredientes que no van a
poder utilizarse en los platos de la recomendación, ya sea porque los comensales tendrían problemas o bien no estén disponibles en la
época del año en la que se celebrará el evento. Por supuesto, esta lista también incluye aquellos ingredientes que el cliente haya
prohibido directamente. Esto resulta muy útil para filtrar los platos en su módulo correspondiente.
\item[Contexto] Contiene información sobre el entorno del evento. Sólo incluye la estación del año en la que se realizará el evento.
\end{description}

% !TEX encoding = UTF-8 Unicode
% !TEX root = ../../report.tex
% 

\subsection{Módulo de platos}
Una vez se han obtenido las preferencias del cliente y el contexto del evento que se quiere realizar, se activa el
\strong{módulo de platos}.
El objetivo del \strong{módulo de platos} es seleccionar los mejores platos disponibles para celebrar el evento que el cliente ha 
descrito en el módulo de abstracción teniendo en cuenta todas sus preferencias. Este módulo está dividido en tres fases o submódulos: 
\strong{filtración}, \strong{puntuación} y \strong{selección}.

\subsubsection{Fase de filtración}
En esta primera fase se crea una \texttt{Recomendacion} por cada plato presente en la ontología. Una \texttt{Recomendacion} no es más
que un contenedor del plato en sí, más una puntuación y sus justificaciones.
A continuación, se descartan todas aquellas recomendaciones cuyos platos son incompatibles con los datos obtenidos del cliente. Más
exactamente, se eliminan las recomendaciones cuyos platos:

\begin{enumerate}
\item Contienen ingredientes prohibidos (ver apartado \ref{abstraccion-datos}).
\item Son exclusivos de un tipo de evento diferente al que se va a celebrar.
\item Son demasiado caros para las limitaciones de precio establecidas.
\item Son para los paladares más exigentes y el cliente no es sibarita.
\end{enumerate}

\subsubsection{Fase de puntuación}
Esta fase consiste en puntuar todas las recomendaciones en función de las preferencias del cliente. Cuando una recomendación
es puntuada, al mismo tiempo se le añade una justificación. De esta manera es posible ofrecer al cliente una explicación detallada
de las recomendaciones que ha recibido.

Sea $p$ un plato de una recomendación, se puntúa (por orden de prioridad):

\begin{description}
\item[Exclusivo del evento] Positivamente si $p$ es exclusivo del tipo de evento a celebrar.
\item[Recomendado para el evento] Positivamente si $p$ es recomendado para el tipo de evento a celebrar.
\item[Tipo de cocina] Positivamente si $p$ es de un tipo de cocina preferido por el cliente y negativamente si no lo es.
\item[Zonas geográficas] Positivamente si $p$ es típico de alguna de las zonas geográficas preferidas por el cliente.
\item[Sibarita] Positivamente si $p$ es para paladares exigentes y el cliente es sibarita.
\item[Temperatura] Si el cliente tiene preferencia de temperatura, positivamente si $p$ se sirve a la temperatura preferida por el cliente
y negativamente en caso contrario.
\item[Dificultad] Negativamente si $p$ supera la dificultad máxima establecida para el número de comensales del evento.
\item[Plato caliente en verano] Negativamente si $p$ se sirve caliente y el evento se celebrará en la estación de verano.
\item[Plato caliente en invierno] Positivamente si $p$ se sirve caliente y el evento se celebrará en la estación de invierno.
\end{description}

Cada aspecto del plato puntuado tiene un peso asignado que permite modificar la prioridad de unas características frente 
otras\footnote{De hecho, los pesos se encuentran centralizados en un \texttt{template} \texttt{Pesos} dentro del propio módulo de
puntuación.}. Hemos decidido darle prioridad a los platos \strong{exclusivos del evento} para asegurarnos que platos como, por ejemplo,
un pastel de boda siempre estén presentes en los menús recomendados para dichos eventos. Luego, hemos considerado más importante
que un plato sea del tipo de cocina preferido por el cliente antes que de una zona geográfica preferida o que se sirva a la
temperatura preferida\footnote{Una posible mejora del sistema sería preguntarle al cliente la prioridad que quiere dar a cada una de
sus preferencias y así obtener menús todavía más personalizados.}...
% !TEX encoding = UTF-8 Unicode
% !TEX root = ../../report.tex
% 

\subsection{Módulo de menús}
Una vez se han seleccionado los hasta 30 primeros platos, 30 segundos y 15
postres, llega el momento de empezar a combinarlos para crear los menús.

El módulo de menús está dividido en tres fases (tres módulos de \texttt{CLIPS})
y se encarga de generar, puntuar y seleccionar los menús abstractos que pasarán
al módulo de vinos. A continuación podremos ver los detalles de su
funcionamiento.

Originalmente hacíamos las combinaciones con los vinos directamente, sin usar
la clase \clase{MenuAbstracto} con \clase{ColorVino}. El problema es que se
producía una explosión combinatoria muy grande sin necesidad, ya que
originalmente el sistema no usaba nada más que el color del vino para valorar
la combinación del vino con los platos.

\subsubsection{Fase de filtración}
Antes que nada, una vez se tienen las instancias de \clase{Recomendacion} para
primeros, segundos y postres, se realizan todas las combinaciones posibles
junto con instancias de \clase{ColorVino}, si el cliente así lo ha
solicitado. Todas estas combinaciones se guardan como instancias de
\clase{MenuAbstracto}.

Hay una serie de menús que querríamos evitar, porque no interesa que estén en
el sistema. Si entre los platos de \strong{primero}, \strong{segundo} o
\strong{postre} que hay en la combinación que se está estudiando hay alguno
repetido, se desecha la combinación. Así que en la única regla importante la
fase, \regla{generar-menus}, se tiene en cuenta y ya directamente no se crean
dichas instancias de \clase{MenuAbstracto}.

Por otro lado, hay que tener en cuenta si el cliente ha pedido recomendaciones
de vino y, además, cuántas (una para el menú o una por plato). Por cada vino
solicitado, y por cada combinación de colores preferidos por el cliente, se
crea un menú abstracto.

Con esto, llegamos a la conclusión de que, en función de la cantidad de vinos
solicitados y los colores favoritos, podemos generar la siguiente cantidad de
menús abstractos (considerando $c$ la cantidad de colores de vino que acepta el cliente, siendo el mínimo 1 y el máximo 3):
\begin{itemize}
\item Si no se ha pedido vino, un total de como máximo $30 \times 30 \times 15
  = 13500$ menús abstractos;
\item si se ha pedido un vino, $13500 \times c$ menús abstractos;
\item y si se ha solicitado un vino por plato, $13500 \times c^2$.
\end{itemize}

Por tanto, como máximo habrá $13500 \times 9 = 121500$ menús abstractos al
acabar la fase. En la práctica habrá menos, ya que muchos platos de la
ontología pueden ser tanto primeros como segundos.

\subsubsection{Fase de puntuación}
Cuando ya se han generado todos los menús posibles, llega el momento de
puntuarlos para decidir cuáles son mejores. Para ello, definimos un
\verb+template+ (hecho no ordenado) \clase{Pesos} con la importancia del
resultado de cada regla.

Lo esencial es crear la puntuación básica del menú, y eso lo hace la regla
\regla{puntuar-menu-platos} sumando los puntos de las recomendaciones que están
como primero, segundo y postre de la instancia de \clase{MenuAbstracto} que se
está tratando.

Luego, cuando ya hay una primera cualificación del menú, \strong{se valora
  negativamente} que el primer y el segundo plato sean \strong{ambos ligeros} o
\strong{pesados}. En el primer caso, los comensales pueden no saciarse, y en el
segundo caso es por razones de salud y ahorro. 

También es necesario \strong{valorar la combinación del vino con los
  platos}. Si solamente hay una posibilidad de vino, se valora positivamente
que ambos platos encajen bien con su color. Si el cliente ha solicitado dos, se
valora que cada color de la combinación que hay en el \clase{MenuAbstracto} en
cuestión sea correcto para su plato respectivo. Es decir, el primer color con
el primer plato y el segundo color con el segundo plato.

Durante las pruebas del sistema nos dimos cuenta de que \strong{en ocasiones se
  ofrecían platos con precios muy dispares}. Si se propone un menú con un plato
que es considerablemente más caro que el otro (sin tener en cuenta el postre),
el cliente puede pensar que el menú no se ha elaborado con el suficiente
cuidado y no verlo correcto. Por ese motivo, cuando la diferencia de precio en
valor absoluto y con redondeo al entero más próximo por defecto es mayor a 1,
se resta una cantidad (5 en nuestro caso) por unidad de diferencia.

No pasa nada porque todas las combinaciones reciban alguna puntuación
negativa. En el caso de la diferencia de precios, solamente se utiliza para
afinar las recomendaciones de menús. Si una es mucho mejor que el resto, no se
verá afectada en exceso.

\subsubsection{Fase de selección}
Cuando los menús abstractos están ya todos puntuados, hay que seleccionar las
combinaciones que pasarán al módulo de vinos. No es útil utilizarlas todas
porque, para empezar, la cantidad puede ser excesiva al combinarlas con las
instancias de \clase{Vino} de la ontología y, por otro lado, muchas de las
combinaciones en los menús abstractos pueden ser realmente malas en comparación
con las que tienen mayor puntuación.

El sistema escoge los 200 mejores menús abstractos. Es un valor que se puede
cambiar, pero es un valor que ha dado unos buenos resultados de tiempo de
generación.

% !TEX encoding = UTF-8 Unicode
% !TEX root = ../../report.tex
% 

\subsection{Módulo de vinos}
Las instancias de \clase{MenuAbstracto} ya podrían servir para presentar las
propuestas de menú al usuario. Sin embargo, falta incluir los vinos concretos
si es el caso. Este módulo se encarga de eso, justamente, además de escoger los
hasta tres menús (uno barato, uno con coste medio, y uno más caro) que se
presentarán al usuario como mejores.

Para ello, se utilizan la clase \clase{Menu} y el \verb+deftemplate+
\clase{SeleccionMenus} como veremos a continuación.

\subsubsection{Fase de filtración}
Esta fase se encarga de recoger los menús abstractos que han pasado la fase de
selección del módulo de menús y combinarlos con los vinos según los colores que
haya establecidos. Además de eso, calcula el precio que tendría el menú con los
dos platos, el postre y, si es el caso, los vinos. Para el caso del vino, el
precio individual se calcula dividiendo entre 4 el precio de la botella.

La cantidad de menús que se generarán en esta fase puede ser:
\begin{itemize}
\item Si no se ha solicitado ningún vino, hasta $200$ (los mismos que menús
  abstractos);
\item si se ha solicitado un vino para el menú, como máximo $200 \times v$, con
  $v$ la cantidad de vinos que hay del color aceptado por cada menú abstracto;
\item si se ha solicitado un vino por plato, como máximo $200 \times v_1 \times
  v_2$, con los $v_i$ la cantidad de vinos de los colores por cada menú
  abstracto para el primer y segundo plato, respectivamente.
\end{itemize}

Del color que más hay es el tinto, con $12$ de las $34$ instancias. En el peor
caso, el cliente habrá solicitado un vino por plato y los 200 menús tendrán el
color tinto para ambos vinos. Como el segundo vino y el primero no pueden ser
iguales, porque el cliente ha solicitado dos vinos distintos, nos encontramos
con una cota superior de $200 \times 12 \times 11 = 26400$ menús al salir de la
fase.

\subsubsection{Fase de puntuación}
En esta fase se puntúa la combinación de los vinos con los platos. Hay que
matizar que solamente es en precio, ya que la puntuación de la corrección del
color se había hecho durante la fase de puntuación del módulo de menús.

La consideración que hemos tomado es que \strong{el vino no puede ser ni
  demasiado caro ni demasiado barato} en comparación con los platos. De forma
similar a cuando los platos tienen precios muy dispares, si el cliente obtiene
una recomendación de un plato de bajo precio con un vino muy caro, o al revés,
puede pensar que algo raro pasa con el sistema.

Consideramos que un vino es caro cuando el precio de un vaso es más caro que el
precio del plato con el que se tomará. Dicho de otro modo, si cuatro platos
cuestan menos que la botella. En ese caso, se da una puntuación negativa de 100
por cada plato del menú con los que se tomará el vino (el primero, el segundo,
o los dos según corresponda) y este último sea \emph{demasiado caro}.

De forma similar, consideramos que un vino es barato cuando (casi) tres
botellas cuestan menos que un plato con el que se tomará. En este caso, se
restan 50 puntos por cada plato con los que se tomará el vino y el último sea
\emph{demasiado barato}.

\subsubsection{Fase de selección}
Cuando ya están todos los posibles menús punutados, hay que seleccionar los
hasta tres que se presentarán al usuario (uno más barato, uno de precio
intermedio y uno más caro). Para ello, inicialmente se calculan los intervalos
de precio para los que un menú se considera barato, medio o caro.

Si el cliente no ha decidido indicar sus propios límites mínimo y máximo de
precio, se toman $0\,€$ como el mínimo y $45\,€$ como el máximo. El cálculo se
realiza dividiendo el intervalo en tres intervalos iguales. Estos cálculos se
hacen en las reglas \regla{actualizar-franja-min},
\regla{actualizar-franja-max} y \regla{actualizar-franjas-intermedias} y se
almacena su resultado en el hecho no ordenado \clase{FranjasPrecio}.

Ya en el momento de seleccionar el menú, lo que hacemos es escoger \strong{el
  menú con mejor puntuación} para \strong{cada intervalo de precios}. Si dos
menús tienen la misma puntuación y caen en el mismo rango de precios,
simplemente se toma una decisión al azar.

Cuando ha acabado esta fase, solamente hace falta presentar los resultados al
usuario.

% !TEX encoding = UTF-8 Unicode
% !TEX root = ../../report.tex
% 

\subsection{Módulo de presentación}
Es importante presentarle los resultados al usuario, ya que el motivo de haber
implementado todo el sistema es crear los menús para el usuario. Este módulo se
encarga de formatear los hasta tres menús guardados en la plantilla
\clase{SeleccionMenus} para mostrar el primer plato, el segundo, el postre y,
si es el caso, los vinos junto con los precios. Además, \strong{pregunta al
  usuario} si quiere ver \strong{las justificaciones}.


\end{document}
