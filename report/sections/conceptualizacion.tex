% !TEX encoding = UTF-8 Unicode
% !TEX root = ../report.tex
% 

\section{Conceptualización del problema}
\subsection{Conceptos principales del dominio}

Después de haber hecho un trabajo iterativo, los conceptos que se modelan en el
dominio de conocimiento son
\begin{enumerate}
  \item Cosas elaboradas, como platos y vino (tienen su nombre y precio).
  \item Dentro de los platos, si son pesados o ligeros y su dificultad de
    preparación.
  \item Los ingredientes que forman parte de los platos y su disponibilidad
    durante las cuatro estaciones del año.
  \item Grandes grupos de comensales, que modelan el tipo de ingredientes que
    no pueden comer.
  \item Los tipos de platos, que contienen información sobre los vinos que
    pueden ir mejor con ellos.
  \item Los eventos a celebrar, que tienen los platos que son propios (o
    recomendables) y la importancia de éstos últimos a la hora de elaborar el
    los menús para el cliente.
  \item Las regiones de procedencia de los platos.
  \item El estilo de los platos. Están los genéricos, que no llegarían a ser
    tradicionales porque son simples platos, los tradicionales (podrían
    considerarse, en parte, folclóricos), y platos modernos. Además, también se
    hace distinción de los platos para sibaritas, que son para los paladares
    más finos, pero que pueden pertenecer a cualquiera de las categorías
    anteriores.
\end{enumerate}

Como hemos indicado anteriormente, los comensales que no beben vino acostumbran
a preferir una bebida concreta de forma individualizada. Por esta razón, hemos
pensado que no es algo a modelar en nuestro dominio de conocimiento.

Por otro lado, en el dominio de solución disponemos de
\begin{enumerate}
  \item Recomendaciones concretas de platos, con los motivos para su
    recomendación y una valoración.
  \item Posibles menús abstractos, que contienen el orden de platos y los
    colores de los vinos más aptos para éstos.
  \item Los menús finales con los platos y vinos concretos, además de las
    razones y valoración de su recomendación.
\end{enumerate}
