% !TEX encoding = UTF-8 Unicode
% !TEX root = ../report.tex
% 

\section{Conceptualización del problema}
\subsection{Conceptos principales del dominio}
Después de haber hecho un trabajo iterativo, los conceptos que se modelan en el
dominio de conocimiento son
\begin{enumerate}
  \item Cosas elaboradas, como platos y vino (tienen su nombre y precio).
  \item Dentro de los platos, si son pesados o ligeros y su dificultad de
    preparación.
  \item Los ingredientes que forman parte de los platos y su disponibilidad
    durante las cuatro estaciones del año.
  \item Grandes grupos de comensales, que modelan el tipo de ingredientes que
    no pueden comer.
  \item Los tipos de platos, que contienen información sobre los vinos que
    pueden ir mejor con ellos.
  \item Los eventos a celebrar, que tienen los platos que son propios (o
    recomendables) y la importancia de éstos últimos a la hora de elaborar el
    los menús para el cliente.
  \item Las regiones de procedencia de los platos.
  \item El estilo de los platos. Están los genéricos, que no llegarían a ser
    tradicionales porque son simples platos, los tradicionales (podrían
    considerarse, en parte, folclóricos), y platos modernos. Además, también se
    hace distinción de los platos para sibaritas, que son para los paladares
    más finos, pero que pueden pertenecer a cualquiera de las categorías
    anteriores.
\end{enumerate}

Como hemos indicado anteriormente, los comensales que no beben vino acostumbran
a preferir una bebida concreta de forma individualizada. Por esta razón, hemos
pensado que no es algo a modelar en nuestro dominio de conocimiento.

Por otro lado, en el dominio de solución disponemos de
\begin{enumerate}
  \item Recomendaciones concretas de platos, con los motivos para su
    recomendación y una valoración.
  \item Posibles menús abstractos, que contienen el orden de platos y los
    colores de los vinos más aptos para éstos.
  \item Los menús finales con los platos y vinos concretos, además de las
    razones y valoración de su recomendación.
\end{enumerate}

\subsection{Problemas y subproblemas}
Como se ha comentado, el problema principal que queremos resolver es el de
proponer y presentar hasta tres menús a un cliente que quiere celebrar un
evento en base a sus restricciones. Para ello hemos dividido el problema en los
subproblemas más pequeños que se detallan a continuación, y que se resuelven en
orden.

\begin{description}
  \item[Abstracción] Se trata de conseguir las preferencias y características
    del evento al usuario.
  \item[Subproblema de selección de platos] Este subproblema se puede dividir
    en tres partes:
    \begin{enumerate}
      \item Filtrar y eliminar los platos que no coinciden con lo solicitado
        por el cliente        
      \item Valorar y puntuar la calidad de los platos en base a las
        preferencias del cliente
      \item Hacer una selección de los primeros, segundos y postres que son
        candidatos a combinarse para crear menús.
    \end{enumerate}
  \item[Subproblema de selección de menús (abstractos)] Aquí se trata de, a
    partir de las posibles combinaciones de platos, encontrar los mejores menús
    junto con el tipo de vino, si es el caso, que queda mejor. Los pasos son
    similares a los usados para el subproblema de selección de platos, con la
    diferencia de que en el paso de puntuación se valoran las combinaciones de
    platos y, dado el caso, tipos de vinos.
  \item[Subproblema de selección de vinos] Una vez se tienen los mejores menús,
    con el tipo de vino, es importante seleccionar un buen vino acorde con los
    tipos y con los platos. Los tres pasos principales son los mismos otra
    vez. Se filtran y eliminan los vinos no acordes con los menús, se puntúa lo
    bien que funciona el vino con el menú (en particular, que el precio sea
    correcto). Finalmente se seleccionan los hasta tres mejores menús que
    entren dentro del rango de precios precisado.
\end{description}
