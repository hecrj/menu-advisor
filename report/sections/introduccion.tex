% !TEX encoding = UTF-8 Unicode
% !TEX root = ../report.tex
% 

\section{Introducción}
\subsection{Descripción del problema}

\begin{wrapfigure}{O}{0.3\textwidth}
  \vspace{-20pt}
  \begin{center}
    \includegraphics[width=0.28\textwidth]{figures/ricorico}
  \end{center}
  \vspace{-20pt}
\end{wrapfigure}

La empresa de catering \emph{Rico Rico} quiere mejorar su eficiencia a la hora
de proponer menús para las celebraciones de sus clientes. Con ese fin, ha
encargado la implementación de un sistema experto basado en el conocimiento que
tienen tras su larga experiencia en el sector. Gracias al sistema, solamente es
necesario que el cliente indique sus preferencias y restricciones concretas
para obtener las recomendaciones más acordes con sus necesidades.

El sistema se encarga de hacer preguntas al cliente para recomendar hasta tres
menús con distintos precios, siempre dentro de los límites exigidos por el
cliente. Entre otras cosas, se precisa saber el tipo de evento, el número de
comensales, los posibles trastornos o preferencias alimenticias (alergias,
razones culturales, ingredientes que no se quieren, etc.), preferencias de
origen geográfico de los platos...

También debe ser capaz de recomendar vinos si el cliente así lo desea, y hacer
un menú lo suficientemente acorde con éstos. Inicialmente se tenía previsto que
las bebidas también dependieran de los platos y el cliente, pero fuera del
terreno del vino los comensales prefieren escoger individualmente la bebida.

La base del conocimiento incluye todos los platos que \emph{Rico Rico} sabe
cocinar, con sus principales ingredientes y disponibilidad por temporadas, su
precio, tipo de plato, dificultad de preparación, información sobre alergias y
problemas alimentarios de razón cultural...

Las distintas botellas de vino también están en el sistema, con su precio,
aunque en el menú a cada comensal se le cobra solamente una parte proporcional
del coste.

La intención es dar hasta tres menús, con tres costes distintos, que entren
dentro de los límites establecidos por el cliente.

\subsection{Análisis del problema}
Este es un problema que si se quiere modelar correctamente precisa de un
conocimiento muy detallado de la cocina. En este caso nos ha faltado el experto
humano que nos asesorara a la hora de valorar qué es mejor y peor.

Aplicar un sistema basado en el conocimiento para resolver el problema parece
una de las mejores maneras. Usualmente, si en una empresa de catering quieren
tener en cuenta todas las restricciones que impone el cliente sin que se
calidad en la propuesta de menús es necesario que quien asesora al cliente
tenga mucha memoria y pueda recordar bien buenas combinaciones de platos. Con
un sistema como este se puede intentar concretar las reglas que dicen cuándo un
menú es bueno para podérselo proponer al cliente. O incluso cuándo es el mejor.
