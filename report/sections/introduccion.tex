% !TEX encoding = UTF-8 Unicode
% !TEX root = ../report.tex
% 

\section{Introducción}
\subsection{Descripción del problema}

\begin{wrapfigure}{O}{0.3\textwidth}
  \vspace{-20pt}
  \begin{center}
    \includegraphics[width=0.28\textwidth]{figures/ricorico}
  \end{center}
  \vspace{-20pt}
\end{wrapfigure}

La empresa de catering \emph{Rico Rico} quiere mejorar su eficiencia a la hora
de proponer menús para las celebraciones de sus clientes. Con ese fin, ha
encargado la implementación de un sistema experto basado en el conocimiento que
tienen tras su larga experiencia en el sector. Gracias al sistema, solamente es
necesario que el cliente indique sus preferencias y restricciones concretas
para obtener las recomendaciones más acordes con sus necesidades.

El sistema se encarga de hacer preguntas al cliente para recomendar hasta tres
menús con distintos precios, siempre dentro de los límites exigidos por el
cliente. Entre otras cosas, se precisa saber el tipo de evento, el número de
comensales, los posibles trastornos o preferencias alimenticias (alergias,
razones culturales, etc.), preferencias de origen cultural de los platos...

También debe ser capaz de recomendar vinos si el cliente así lo desea, y hacer
un menú lo suficientemente acorde con los éstos. Inicialmente se tenía previsto
que las bebidas también dependieran de los platos y el cliente, pero fuera del
terreno del vino los comensales prefieren escoger individualmente la bebida.
