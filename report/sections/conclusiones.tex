% !TEX encoding = UTF-8 Unicode
% !TEX root = ../report.tex
% 

\section{Conclusiones y valoración}
Ha sido un trabajo complejo. Encontramos que quizá nos falta todavía más
variedad en los platos, especialmente aquellos caros. Por otro lado, es difícil
rellenar una ontología así y, más todavía, cuando la cocina es un campo tan
extenso del que no somos expertos ni mucho menos.

A la hora de realizar las pruebas finales, y en algún caso algo antes pero ya
cuando habíamos decidido dar el proyecto por finalizado para cambios mayores,
hemos visto que hay algunos casos que habría estado bien tener en cuenta. Por
ejemplo, en un evento donde hay gente musulmana no preguntar si se quiere vino,
ya que no tiene sentido.

También habría sido interesante implementar un sistema distinto para la hora de
indicar los ingredientes no deseados, que leyera la entrada del usuario, pues
la lista de ingredientes es muy extensa y se puede hacer pesado leerla.

Finalmente, una buena mejora podría ser indicar platos buenos para determinados
tipos de comensales, no solamente los prohibidos (por ejemplo, platos
especiales para niños). En ningún momento se nos ha pasado por la mente hasta
ahora.

Afortunadamente, \strong{el sistema es lo suficientemente modular} como para
poder ampliarlo en un futuro próximo sin mayores problemas, y creemos que en
ese sentido tenemos un buen trabajo.
