% !TEX encoding = UTF-8 Unicode
% !TEX root = ../report.tex
% 

\section{Bodas}

\subsection{Con bastantes comensales, problemas alérgicos}
Este caso representa una boda con bastantes comensales, unos 60. Se quieren
platos modernos, a poder ser de países orientales, y aptos para los paladares
más finos. Además, hay problemas porque hay comensales con celiaquía y alergia
al huevo, por lo que lo mejor es evitar según qué ingredientes. Además, se
precisa comida caliente porque la celebración será a finales de otoño.

Como la boda es de una familia rica, el límite de precios es bastante amplio,
pero elevado: entre 30 y 90 euros.

El resultado de la ejecución es
\lstinputlisting{../test/boda-muchos-problemas.txt}

\section{Comidas familiares}
\subsection{Caso general}
En este caso se representa una comida familiar con 10 comensales realizada en invierno y sin vinos. No se especifica ninguna otra
preferencia por parte del cliente.

El resultado es el siguiente:
\lstinputlisting{../test/comida-familiar-general.txt}

La propuesta más cara no está disponible, ya que el cliente ha decidido que no se le recomienden platos exclusivos, que son los más
caros. En las otras propuestas se observa una valoración positiva a los platos calientes, ya que el evento se produce en invierno,
y a los platos recomendados para comidas familiares.

\section{Comuniones}
\subsection{Comunión tradicional española}
En este caso se representa una comunión de 40 comensales, a la que se le quieren dar prioridad los platos tradicionales y genéricos
típicos de España. La comunión se realizará en invierno y los menús recomendados deben tener un precio entre 10 y 40 euros.

El resultado es el siguiente:
\lstinputlisting{../test/comunion-espana.txt}

En este caso se ofrecen tres menús, dentro de las limitaciones de precio. Se observa que el pastel de comunión está presente en
todos los menús propuestos debido a que es un \strong{plato exclusivo para comuniones}. Además, también se observa la presencia de
\strong{platos españoles genéricos y tradicionales} en \strong{todos} los menús. Por último, y como en anteriores casos, los platos
recomendados suelen ser \strong{calientes} ya que apetecen más en invierno.
