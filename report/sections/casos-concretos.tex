% !TEX encoding = UTF-8 Unicode
% !TEX root = ../report.tex
% 

\section{Bodas}

\subsection{Con bastantes comensales, problemas alérgicos}
Este caso representa una boda con bastantes comensales, unos 60. Se quieren
platos modernos, a poder ser de países orientales, y aptos para los paladares
más finos. Además, hay problemas porque hay comensales con celiaquía y alergia
al huevo, por lo que lo mejor es evitar según qué ingredientes. Además, se
precisa comida caliente porque la celebración será a finales de otoño.

Como la boda es de una familia rica, el límite de precios es bastante amplio,
pero elevado: entre 30 y 90 euros.

El resultado de la ejecución es
\lstinputlisting{../test/boda-muchos-problemas.txt}

Puede observarse que no nos ha dado una solución con precio más elevado. La
razón más probable es que en la ontología faltan platos caros. Por otro lado,
no ha propuesto ningún plato de las regiones que hemos solicitado, pero sí que
son modernos y calientes.

La razón más probable para la falta de platos regionales es que en muchas
ocasiones, los platos orientales contienen algún tipo de harina de trigo que no
es apta para celíacos. También está la restricción de la alergia al huevo, que
no es pequeña.

Por otro lado, es muy fácil ver que la única diferencia entre ambos menús son
los vinos. En las pruebas realizadas durante el desarrollo nos había pasado,
pero tampoco habíamos probado el caso de poner unos límites de precio tan
altos. El hecho de dar un precio mínimo tan elevado seguramente ha provocado
que solamente se aceptaran los menús con los platos más caros, que resultan ser
el plato de camarones y el solomillo.

\subsection{Una boda de gente sencilla y vegetariana}
En esta boda, habrá pocos comensales (unos 20, familiares y amigos). Se celebra
en verano, y se prefieren platos vegetarianos, pero sin tofu. Además, no se
quiere nada de vino y que el coste no suba demasiado (como máximo 30 euros por
persona). Los platos pueden ser de cualquier tipo, ya que a menudo les cuesta
encontrar comida vegetariana ``estándar'', pero nada de platos para sibaritas.

Se puede ver el resultado a continuación:
\lstinputlisting{../test/boda-pocos-vegetarianos-sin-tofu.txt}

En realidad se han indicado pocas preferencias, por eso el resultado es tan
``malo'' (con pocas justificaciones). Por otro lado, el hecho de pedir comida
vegetariana sin tofu puede haber provocado un gran descenso en la cantidad de
platos disponibles.

La poca variedad de platos y el punto del precio han provocado que el menú más
barato (muy barato para ser de boda, por otro lado), no incluya pastel sino
melón.

\section{Comidas familiares}
\subsection{Caso general}
En este caso se representa una comida familiar con 10 comensales realizada en invierno y sin vinos. No se especifica ninguna otra
preferencia por parte del cliente.

El resultado es el siguiente:
\lstinputlisting{../test/comida-familiar-general.txt}

La propuesta más cara no está disponible, ya que el cliente ha decidido que no se le recomienden platos exclusivos, que son los más
caros. En las otras propuestas se observa una valoración positiva a los platos calientes, ya que el evento se produce en invierno,
y a los platos recomendados para comidas familiares.

\section{Comuniones}
\subsection{Comunión tradicional española}
En este caso se representa una comunión de 40 comensales, a la que se le quieren dar prioridad los platos tradicionales y genéricos
típicos de España. La comunión se realizará en invierno y los menús recomendados deben tener un precio entre 10 y 40 euros.

El resultado es el siguiente:
\lstinputlisting{../test/comunion-espana.txt}

En este caso se ofrecen tres menús, dentro de las limitaciones de precio. Se observa que el pastel de comunión está presente en
todos los menús propuestos debido a que es un \strong{plato exclusivo para comuniones}. Además, también se observa la presencia de
\strong{platos españoles genéricos y tradicionales} en \strong{todos} los menús. Por último, y como en anteriores casos, los platos
recomendados suelen ser \strong{calientes} ya que apetecen más en invierno.
