% !TEX encoding = UTF-8 Unicode
% !TEX root = ../../report.tex
% 

\subsection{Módulo de abstracción de datos}
El módulo de abstracción de datos se encarga de obtener las \strong{preferencias del cliente} y abstraerlas para que el sistema experto
pueda utilizarlas. Este módulo es el primero en ejecutarse, ya que es necesario conocer primero las preferencias del cliente
antes de recomendarle nada.

\subsubsection{Los datos}
\label{abstraccion-datos}
Veamos con detalle las interacciones necesarias por parte del cliente para introducir los datos que le pide el módulo de abstracción
y que el experto desarrollado tendrá en cuenta para la recomendación:

\begin{description}
\item[Evento] Indicar de entre los tipos de evento disponibles, aquel que desee realizar.
\item[Número de comensales] Introducir el número de comensales del evento.
\item[Tipos de cocina preferidos] Escoger de entre los tipos de cocina disponibles, aquellos que prefiere sobre los demás.
\item[Sibarita] Indicar si es sibarita y prefiere platos para los paladares más exigentes.
\item[Regiones preferidas] Indicar si desea dar prioridad a los platos de una zona geográfica concreta y, en caso afirmativo, seleccionar
las zonas preferidas de entre las disponibles.
\item[Tipos de comensal] Seleccionar los tipos de comensales que asistirán al evento (vegetarianos, alérgicos, etc.).
\item[Temperatura preferida] Indicar si prefiere platos en una temperatura concreta y, en caso afirmativo, indicar la temperatura preferida de entre las disponibles.
\item[Estación del año] Seleccionar la estación del año en la que se celebrará el evento.
\item[Ingredientes prohibidos] Indicar si desea prohibir algunos ingredientes en concreto y, en caso afirmativo, seleccionar los
ingredientes prohibidos de entre los disponibles.
\item[Limitaciones de precio] Indicar si desea establecer límites de precio y, en caso afirmativo, introducir el precio máximo y
mínimo para los menús recomendados.
\item[Bebida] Escoger la bebida que desea: un vino para el menú, un vino para cada plato o otra bebida.
\item[Colores de vino preferidos] En caso de que se haya preferido vino para la bebida, seleccionar los colores de vino preferidos de
entre los disponibles.
\end{description}

\subsubsection{Abstracción}
Para abstraer los datos y representarlos en \texttt{CLIPS} hemos utilizado dos \texttt{template}s:

\begin{description}
\item[Preferencias] Cada uno de sus \texttt{slots} representa una preferencia del cliente. Lo más destacable es que se ha utilizado un
 \texttt{multislot} \texttt{ingredientes-prohibidos} que se llena durante la abstracción con aquellos ingredientes que no van a
poder utilizarse en los platos de la recomendación, ya sea porque los comensales tendrían problemas o bien no estén disponibles en la
época del año en la que se celebrará el evento. Por supuesto, esta lista también incluye aquellos ingredientes que el cliente haya
prohibido directamente. Esto resulta muy útil para filtrar los platos en su módulo correspondiente.
\item[Contexto] Contiene información sobre el entorno del evento. Sólo incluye la estación del año en la que se realizará el evento.
\end{description}
