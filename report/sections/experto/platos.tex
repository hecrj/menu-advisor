% !TEX encoding = UTF-8 Unicode
% !TEX root = ../../report.tex
% 

\subsection{Módulo de platos}
Una vez se han obtenido las preferencias del cliente y el contexto del evento que se quiere realizar, se activa el
\strong{módulo de platos}.
El objetivo del \strong{módulo de platos} es seleccionar los mejores platos disponibles para celebrar el evento que el cliente ha 
descrito en el módulo de abstracción teniendo en cuenta todas sus preferencias. Este módulo está dividido en tres fases o submódulos: 
\strong{filtración}, \strong{puntuación} y \strong{selección}.

\subsubsection{Fase de filtración}
En esta primera fase se crea una \texttt{Recomendacion} por cada plato presente en la ontología. Una \texttt{Recomendacion} no es más
que un contenedor del plato en sí, más una puntuación y sus justificaciones.
A continuación, se descartan todas aquellas recomendaciones cuyos platos son incompatibles con los datos obtenidos del cliente. Más
exactamente, se eliminan las recomendaciones cuyos platos:

\begin{enumerate}
\item Contienen ingredientes prohibidos (ver apartado \ref{abstraccion-datos}).
\item Son exclusivos de un tipo de evento diferente al que se va a celebrar.
\item Son demasiado caros para las limitaciones de precio establecidas.
\item Son para los paladares más exigentes y el cliente no es sibarita.
\end{enumerate}

\subsubsection{Fase de puntuación}
Esta fase consiste en puntuar todas las recomendaciones en función de las preferencias del cliente. Cuando una recomendación
es puntuada, al mismo tiempo se le añade una justificación. De esta manera es posible ofrecer al cliente una explicación detallada
de las recomendaciones que ha recibido.

Sea $p$ un plato de una recomendación, se puntúa (por orden de prioridad):

\begin{description}
\item[Exclusivo del evento] Positivamente si $p$ es exclusivo del tipo de evento a celebrar.
\item[Recomendado para el evento] Positivamente si $p$ es recomendado para el tipo de evento a celebrar.
\item[Tipo de cocina] Positivamente si $p$ es de un tipo de cocina preferido por el cliente y negativamente si no lo es.
\item[Zonas geográficas] Positivamente si $p$ es típico de alguna de las zonas geográficas preferidas por el cliente.
\item[Sibarita] Positivamente si $p$ es para paladares exigentes y el cliente es sibarita.
\item[Temperatura] Si el cliente tiene preferencia de temperatura, positivamente si $p$ se sirve a la temperatura preferida por el cliente
y negativamente en caso contrario.
\item[Dificultad] Negativamente si $p$ supera la dificultad máxima establecida para el número de comensales del evento.
\item[Plato caliente en verano] Negativamente si $p$ se sirve caliente y el evento se celebrará en la estación de verano.
\item[Plato caliente en invierno] Positivamente si $p$ se sirve caliente y el evento se celebrará en la estación de invierno.
\end{description}

Cada aspecto del plato puntuado tiene un peso asignado que permite modificar la prioridad de unas características frente 
otras\footnote{De hecho, los pesos se encuentran centralizados en un \texttt{template} \texttt{Pesos} dentro del propio módulo de
puntuación.}. Hemos decidido darle prioridad a los platos \strong{exclusivos del evento} para asegurarnos de que platos como, por 
ejemplo, un pastel de boda siempre estén presentes en los menús recomendados para dichos eventos. Luego, hemos considerado más
importante que un plato sea del tipo de cocina preferido por el cliente antes que de una zona geográfica preferida o que se sirva a la
temperatura preferida\footnote{Una posible mejora del sistema sería preguntarle al cliente la prioridad que quiere dar a cada una de
sus preferencias y así obtener menús todavía más personalizados.}.

\subsection{Fase de selección}
Una vez los platos ya han sido puntuados se pasa a la \strong{fase de selección}. En esta fase se seleccionan los
\strong{30 mejores primeros}, los \strong{30 mejores segundos} y los \strong{15 mejores postres}. Si no hiciéramos esto el
sistema no sería escalable, ya que el número de menús tiene crecimiento cúbico respecto al número de primeros, segundos y postres.
La cantidad de platos de cada tipo seleccionados\footnote{Se seleccionan menos postres porque es más normal que haya repetición de platos entre los primeros y los segundos.} permite para poder generar menús con la suficiente diversidad sin que se dispare
el número de combinaciones aunque se añadan más platos a la ontología.
