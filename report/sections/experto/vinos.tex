% !TEX encoding = UTF-8 Unicode
% !TEX root = ../../report.tex
% 

\subsection{Módulo de vinos}
Las instancias de \clase{MenuAbstracto} ya podrían servir para presentar las
propuestas de menú al usuario. Sin embargo, falta incluir los vinos concretos
si es el caso. Este módulo se encarga de eso, justamente, además de escoger los
hasta tres menús (uno barato, uno con coste medio, y uno más caro) que se
presentarán al usuario como mejores.

Para ello, se utilizan la clase \clase{Menu} y el \verb+deftemplate+
\clase{SeleccionMenus} como veremos a continuación.

\subsubsection{Fase de filtración}
Esta fase se encarga de recoger los menús abstractos que han pasado la fase de
selección del módulo de menús y combinarlos con los vinos según los colores que
haya establecidos. Además de eso, calcula el precio que tendría el menú con los
dos platos, el postre y, si es el caso, los vinos. Para el caso del vino, el
precio individual se calcula dividiendo entre 4 el precio de la botella.

La cantidad de menús que se generarán en esta fase puede ser:
\begin{itemize}
\item Si no se ha solicitado ningún vino, hasta $200$ (los mismos que menús
  abstractos);
\item si se ha solicitado un vino para el menú, como máximo $200 \times v$, con
  $v$ la cantidad de vinos que hay del color aceptado por cada menú abstracto;
\item si se ha solicitado un vino por plato, como máximo $200 \times v_1 \times
  v_2$, con los $v_i$ la cantidad de vinos de los colores por cada menú
  abstracto para el primer y segundo plato, respectivamente.
\end{itemize}

Del color que más hay es el tinto, con $12$ de las $34$ instancias. En el peor
caso, el cliente habrá solicitado un vino por plato y los 200 menús tendrán el
color tinto para ambos vinos. Como el segundo vino y el primero no pueden ser
iguales, porque el cliente ha solicitado dos vinos distintos, nos encontramos
con una cota superior de $200 \times 12 \times 11 = 26400$ menús al salir de la
fase.

\subsubsection{Fase de puntuación}
En esta fase se puntúa la combinación de los vinos con los platos. Hay que
matizar que solamente es en precio, ya que la puntuación de la corrección del
color se había hecho durante la fase de puntuación del módulo de menús.

La consideración que hemos tomado es que \strong{el vino no puede ser ni
  demasiado caro ni demasiado barato} en comparación con los platos. De forma
similar a cuando los platos tienen precios muy dispares, si el cliente obtiene
una recomendación de un plato de bajo precio con un vino muy caro, o al revés,
puede pensar que algo raro pasa con el sistema.

Consideramos que un vino es caro cuando el precio de un vaso es más caro que el
precio del plato con el que se tomará. Dicho de otro modo, si cuatro platos
cuestan menos que la botella. En ese caso, se da una puntuación negativa de 100
por cada plato del menú con los que se tomará el vino (el primero, el segundo,
o los dos según corresponda) y este último sea \emph{demasiado caro}.

De forma similar, consideramos que un vino es barato cuando (casi) tres
botellas cuestan menos que un plato con el que se tomará. En este caso, se
restan 50 puntos por cada plato con los que se tomará el vino y el último sea
\emph{demasiado barato}.

\subsubsection{Fase de selección}
Cuando ya están todos los posibles menús punutados, hay que seleccionar los
hasta tres que se presentarán al usuario (uno más barato, uno de precio
intermedio y uno más caro). Para ello, inicialmente se calculan los intervalos
de precio para los que un menú se considera barato, medio o caro.

Si el cliente no ha decidido indicar sus propios límites mínimo y máximo de
precio, se toman $0\,€$ como el mínimo y $45\,€$ como el máximo. El cálculo se
realiza dividiendo el intervalo en tres intervalos iguales. Estos cálculos se
hacen en las reglas \regla{actualizar-franja-min},
\regla{actualizar-franja-max} y \regla{actualizar-franjas-intermedias} y se
almacena su resultado en el hecho no ordenado \clase{FranjasPrecio}.

Ya en el momento de seleccionar el menú, lo que hacemos es escoger \strong{el
  menú con mejor puntuación} para \strong{cada intervalo de precios}. Si dos
menús tienen la misma puntuación y caen en el mismo rango de precios,
simplemente se toma una decisión al azar.

Cuando ha acabado esta fase, solamente hace falta presentar los resultados al
usuario.
