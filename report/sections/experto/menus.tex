% !TEX encoding = UTF-8 Unicode
% !TEX root = ../../report.tex
% 

\subsection{Módulo de menús}
Una vez se han seleccionado los hasta 30 primeros platos, 30 segundos y 15
postres, llega el momento de empezar a combinarlos para crear los menús.

El módulo de menús está dividido en tres fases (tres módulos de \texttt{CLIPS})
y se encarga de generar, puntuar y seleccionar los menús abstractos que pasarán
al módulo de vinos. A continuación podremos ver los detalles de su
funcionamiento.

\subsubsection{Fase de filtración}
Antes que nada, una vez se tienen las instancias de \clase{Recomendacion} para
primeros, segundos y postres, se realizan todas las combinaciones posibles
junto con instancias de \clase{ColorVino}, si el cliente así lo ha
solicitado. Todas estas combinaciones se guardan como instancias de
\clase{MenuAbstracto}.

Hay una serie de menús que querríamos evitar, porque no interesa que estén en
el sistema. Si entre los platos de \strong{primero}, \strong{segundo} o
\strong{postre} que hay en la combinación que se está estudiando hay alguno
repetido, se desecha la combinación. Así que en la única regla importante la
fase, \regla{generar-menus}, se tiene en cuenta y ya directamente no se crean
dichas instancias de \clase{MenuAbstracto}.

Por otro lado, hay que tener en cuenta si el cliente ha pedido recomendaciones
de vino y, además, cuántas (una para el menú o una por plato). Por cada vino
solicitado, y por cada combinación de colores preferidos por el cliente, se
crea un menú abstracto.

Con esto, llegamos a la conclusión de que, en función de la cantidad de vinos
solicitados y los colores favoritos, podemos generar la siguiente cantidad de
menús abstractos (considerando $c$ la cantidad de colores de vino que acepta el cliente, siendo el mínimo 1 y el máximo 3):
\begin{itemize}
\item Si no se ha pedido vino, un total de como máximo $30 \times 30 \times 15
  = 13500$ menús abstractos;
\item si se ha pedido un vino, $13500 \times c$ menús abstractos;
\item y si se ha solicitado un vino por plato, $13500 \times c^2$.
\end{itemize}

Por tanto, como máximo habrá $13500 \times 9 = 121500$ menús abstractos al
acabar la fase. En la práctica habrá menos, ya que muchos platos pueden ser
primeros y segundos a la vez.


