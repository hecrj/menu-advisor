% !TEX encoding = UTF-8 Unicode
% !TEX root = ../../report.tex
% 

\subsection{Módulo de menús}
Una vez se han seleccionado los hasta 30 primeros platos, 30 segundos y 15
postres, llega el momento de empezar a combinarlos para crear los menús.

El módulo de menús está dividido en tres fases (tres módulos de \texttt{CLIPS})
y se encarga de generar, puntuar y seleccionar los menús abstractos que pasarán
al módulo de vinos. A continuación podremos ver los detalles de su
funcionamiento.

Originalmente hacíamos las combinaciones con los vinos directamente, sin usar
la clase \clase{MenuAbstracto} con \clase{ColorVino}. El problema es que se
producía una explosión combinatoria muy grande sin necesidad, ya que
originalmente el sistema no usaba nada más que el color del vino para valorar
la combinación del vino con los platos.

\subsubsection{Fase de filtración}
Antes que nada, una vez se tienen las instancias de \clase{Recomendacion} para
primeros, segundos y postres, se realizan todas las combinaciones posibles
junto con instancias de \clase{ColorVino}, si el cliente así lo ha
solicitado. Todas estas combinaciones se guardan como instancias de
\clase{MenuAbstracto}.

Hay una serie de menús que querríamos evitar, porque no interesa que estén en
el sistema. Si entre los platos de \strong{primero}, \strong{segundo} o
\strong{postre} que hay en la combinación que se está estudiando hay alguno
repetido, se desecha la combinación. Así que en la única regla importante la
fase, \regla{generar-menus}, se tiene en cuenta y ya directamente no se crean
dichas instancias de \clase{MenuAbstracto}.

Por otro lado, hay que tener en cuenta si el cliente ha pedido recomendaciones
de vino y, además, cuántas (una para el menú o una por plato). Por cada vino
solicitado, y por cada combinación de colores preferidos por el cliente, se
crea un menú abstracto.

Con esto, llegamos a la conclusión de que, en función de la cantidad de vinos
solicitados y los colores favoritos, podemos generar la siguiente cantidad de
menús abstractos (considerando $c$ la cantidad de colores de vino que acepta el cliente, siendo el mínimo 1 y el máximo 3):
\begin{itemize}
\item Si no se ha pedido vino, un total de como máximo $30 \times 30 \times 15
  = 13500$ menús abstractos;
\item si se ha pedido un vino, $13500 \times c$ menús abstractos;
\item y si se ha solicitado un vino por plato, $13500 \times c^2$.
\end{itemize}

Por tanto, como máximo habrá $13500 \times 9 = 121500$ menús abstractos al
acabar la fase. En la práctica habrá menos, ya que muchos platos de la
ontología pueden ser tanto primeros como segundos.

\subsubsection{Fase de puntuación}
Cuando ya se han generado todos los menús posibles, llega el momento de
puntuarlos para decidir cuáles son mejores. Para ello, creamos un
\emph{deftemplate} (hecho no ordenado) \clase{Pesos} con la importancia del
resultado de cada regla.

Lo esencial es crear la puntuación básica del menú, y eso lo hace la regla
\regla{puntuar-menu-platos} sumando los puntos de las recomendaciones que están
como primero, segundo y postre de la instancia de \clase{MenuAbstracto} que se
está tratando.

Luego, cuando ya hay una primera cualificación del menú, \strong{se valora
  negativamente} que el primer y el segundo plato sean \strong{ambos ligeros} o
\strong{pesados}. En el primer caso, los comensales pueden no saciarse, y en el
segundo caso es por razones de salud y ahorro. 

También es necesario \strong{valorar la combinación del vino con los
  platos}. Si solamente hay una posibilidad de vino, se valora positivamente
que ambos platos encajen bien con su color. Si el cliente ha solicitado dos, se
valora que cada color de la combinación que hay en el \clase{MenuAbstracto} en
cuestión sea correcto para su plato respectivo. Es decir, el primer color con
el primer plato y el segundo color con el segundo plato.

Durante las pruebas del sistema nos dimos cuenta de que \strong{en ocasiones se
  ofrecían platos con precios muy dispares}. Si se propone un menú con un plato
que es considerablemente más caro que el otro (sin tener en cuenta el postre),
el cliente puede pensar que el menú no se ha elaborado con el suficiente
cuidado y no verlo correcto. Por ese motivo, cuando la diferencia de precio en
valor absoluto y con redondeo al entero más próximo por defecto es mayor a 1,
se resta una cantidad (5 en nuestro caso) por unidad de diferencia.

No pasa nada porque todas las combinaciones reciban alguna puntuación
negativa. En el caso de la diferencia de precios, solamente se utiliza para
afinar las recomendaciones de menús. Si una es mucho mejor que el resto, no se
verá afectada en exceso.
